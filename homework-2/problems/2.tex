Find a basis and dimension of the following subspaces.
\begin{enumerate}[label=(\alph*)]
	\item $\mathcal{S}_1 = \left\{ \mathbf{x} =
		      \left[ x_1 \ x_2 \ x_3 \ x_4 \right]^T
		      \in \mathbb{R}^4 \ | \
		      x_1 = -2.8x_2, x_2 + 3x_3 + x_4 = 0
		      \right\}$
	      \begin{tcolorbox}
		      \textbf{Solution:} \par
		      Notice that $\mathcal{S}_1$ is exactly the solution set of the
		      linear equations $x_1 = -2.8x_2, x_2 + 3x_3 + x_4 = 0$. We can
		      rewrite $\mathbf{x}$ as:
		      $$
			      \mathbf{x} =
			      \begin{bmatrix}
				      x_1 \\
				      x_2 \\
				      x_3 \\
				      x_4
			      \end{bmatrix} =
			      x_{2}
			      \begin{bmatrix}
				      -2.8 \\
				      1    \\
				      0    \\
				      -1
			      \end{bmatrix} +
			      x_{3}
			      \begin{bmatrix}
				      0 \\
				      0 \\
				      1 \\
				      -3
			      \end{bmatrix}
		      $$
		      Thus,
		      $$
			      \left\{
			      \begin{bmatrix}
				      -2.8 \\
				      1    \\
				      0    \\
				      -1
			      \end{bmatrix},
			      \begin{bmatrix}
				      0 \\
				      0 \\
				      1 \\
				      -3
			      \end{bmatrix}
			      \right\}
		      $$
		      is a basis of $\mathcal{S}_1$. Since $\mathcal{S}_1$ has a basis
		      with two vectors, it has a dimension of 2.

	      \end{tcolorbox}

	\item $\mathcal{S}_2 = \left\{ \mathbf{X}
		      \in \mathbb{R}^{n \times n} \ | \
		      \operatorname{Trace}(\mathbf{X}) = 0,
		      \mathbf{X}_{1,2} = \mathbf{X}_{2,1} = 0
		      \right\}$
	      \begin{tcolorbox}
		      \textbf{Solution:} \par
		      The trace of a matrix is the sum of its diagonal entries. Thus,
		      for $\operatorname{Trace}(\mathbf{X}) = 0$, we have:
		      $$
			      \sum_{i=1}^{n} \mathbf{X}_{i,i} = 0
		      $$
		      Therefore, $\mathbf{X}_{1,1} = -\sum_{i=2}^{n} \mathbf{X}_{i,i}$,
		      and the remaining terms on the diagonal can be chosen freely.
		      Altogether, we have the following criteria:
		      \begin{enumerate}
			      \item $\mathbf{X}_{1,1} = -\sum_{i=2}^{n} \mathbf{X}_{i,i}$
			      \item $\mathbf{X}_{1,2} = 0$
			      \item $\mathbf{X}_{2,1} = 0$
		      \end{enumerate}
		      Therefore, we can rewrite $\mathbf{X}$ as:
		      $$
			      \begin{aligned}
				      \mathbf{X} & =
				      -\sum_{i=2}^{n} \mathbf{X}_{i,i}
				      \begin{bmatrix}
					      1      & 0 & \cdots & 0      \\
					      0      & 0 &        & 0      \\
					      \vdots &   & \ddots & \vdots \\
					      0      & 0 & \cdots & 0      \\
				      \end{bmatrix}
				      +
				      0
				      \begin{bmatrix}
					      0      & 1 & \cdots & 0      \\
					      0      & 0 &        & 0      \\
					      \vdots &   & \ddots & \vdots \\
					      0      & 0 & \cdots & 0      \\
				      \end{bmatrix}
				      +
				      0
				      \begin{bmatrix}
					      0      & 0 & \cdots & 0      \\
					      1      & 0 &        & 0      \\
					      \vdots &   & \ddots & \vdots \\
					      0      & 0 & \cdots & 0      \\
				      \end{bmatrix} \\
				                 & +
				      \mathbf{X}_{i,j}
				      \begin{bmatrix}
					      0      & 0 & \cdots & 0      \\
					      0      & 0 &        & 0      \\
					      \vdots &   & \ddots & \vdots \\
					      0      & 0 & \cdots & 0      \\
				      \end{bmatrix}
				      +
				      \ldots
				      +
				      \mathbf{X}_{n,n}
				      \begin{bmatrix}
					      0      & 0 & \cdots & 0      \\
					      0      & 0 &        & 0      \\
					      \vdots &   & \ddots & \vdots \\
					      0      & 0 & \cdots & 1      \\
				      \end{bmatrix}
			      \end{aligned}
		      $$
		      Thus, our basis will contain the set of matrices
		      $$
			      \left\{
			      \begin{bmatrix}
				      1      & 0 & \cdots & 0      \\
				      0      & 0 &        & 0      \\
				      \vdots &   & \ddots & \vdots \\
				      0      & 0 & \cdots & 0      \\
			      \end{bmatrix}
			      \cdots
			      \begin{bmatrix}
				      0      & 0 & \cdots & 0      \\
				      0      & 0 &        & 0      \\
				      \vdots &   & \ddots & \vdots \\
				      0      & 0 & \cdots & 1      \\
			      \end{bmatrix}
			      \right\}
		      $$
		      excluding the matrices:
		      $$
			      \left\{
			      \begin{bmatrix}
				      1      & 0 & \cdots & 0      \\
				      0      & 0 &        & 0      \\
				      \vdots &   & \ddots & \vdots \\
				      0      & 0 & \cdots & 0      \\
			      \end{bmatrix},
			      \begin{bmatrix}
				      0      & 1 & \cdots & 0      \\
				      0      & 0 &        & 0      \\
				      \vdots &   & \ddots & \vdots \\
				      0      & 0 & \cdots & 0      \\
			      \end{bmatrix},
			      \begin{bmatrix}
				      0      & 0 & \cdots & 0      \\
				      1      & 0 &        & 0      \\
				      \vdots &   & \ddots & \vdots \\
				      0      & 0 & \cdots & 0      \\
			      \end{bmatrix}
			      \right\}
		      $$

		      Since every other element in $\mathbf{X}$ can be chosen freely,
		      $\mathcal{S}_1$ has a dimension of $n^2 - 3$.
	      \end{tcolorbox}
\end{enumerate}
