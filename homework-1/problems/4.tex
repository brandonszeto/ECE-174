Consider the stacked vectors
$$
	c_1=\left[\begin{array}{l}
			a_1 \\
			b_1
		\end{array}\right], c_2=\left[\begin{array}{l}
			a_2 \\
			b_2
		\end{array}\right], \ldots, c_k=\left[\begin{array}{l}
			a_k \\
			b_k
		\end{array}\right]
$$
where $\mathbf{a}_1, \mathbf{a}_2, \ldots, \mathbf{a}_{\mathbf{k}} \in
	\mathbb{R}^n$,
$\mathbf{b}_1, \mathbf{b}_2, \ldots,
	\mathbf{b}_{\mathbf{k}} \in \mathbb{R}^m$, and $m \leq n$.

\begin{enumerate}[label=(\alph*)]
	\item Suppose $a_1, a_2, \ldots, a_k$ are linearly independent. No
	      assumptions about $b_1, b_2, \ldots, b_k$ are made. Can we conclude that
	      the stacked vectors
	      $c_1, c_2, \ldots, c_k$ are linearly independent?

	      % SOLUTION
	      \par \textbf{Solution:}
	      \par We can conclude that the stacked vectors $c_1, c_2, \ldots, c_k$
	      are also linearly independent. Suppose $\alpha_1 c_1 + \ldots
		      \alpha_kc_k = 0$. Then we have $\alpha_1 a_1 + \ldots \alpha_ka_k = 0$
	      and $\alpha_1 b_1 + \ldots \alpha_kb_k = 0$. Since $a_1 \ldots a_k$ are
	      linearly independent, we must have $\alpha_1 \ldots \alpha_k = 0$.
	      Thus, $c_1 \ldots c_k$ are linearly independent. \qed

	\item Suppose $a_1, a_2, \ldots, a_k$ are linearly dependent. Again, no
	      assumptions about $b_1, b_2, \ldots, b_k$ are made. Can we conclude that
	      the stacked vectors
	      $c_1, c_2, \ldots, c_k$ are linearly independent?

	      % SOLUTION
	      \par \textbf{Solution:}
	      \par Vectors $c_1 \ldots c_k$ are not necessarily linearly independent.
	      In the case that $b_1 \ldots b_k$ are linearly independent, $c_1
		      \ldots c_k$ must also be linearly independent. However, in the case
	      where $b_1 \ldots b_k$ are linearly dependent, there is no guarantee
	      that $c_1 \ldots c_k$ are linearly dependent or independent, and the
	      linear dependence or independence of the stacked vectors $c_1 \ldots
		      c_k$ will rely on the relationship between the vectors $a_1 \ldots
		      a_k$ and $b_1 \ldots b_k$. \qed

	\item What happens in parts (a) and (b) when $m > n$?

	      % SOLUTION
	      \par \textbf{Solution:}
	      \par (a) Nothing will change. As long as one set of vectors is
	      linearly independent, their stacked vectors will be independent.
	      \par (b) In the case where $\mathbf{B}$ is linearly independent,
	      nothing will change. One set of vectors being linearly independent
	      make it such that the stacked vetors will be linearly independent.
	      However, in the case where $\mathbf{B}$ is linearly dependent, a
	      greater number of vectors contributed by $\mathbf{B}$ would correspond
	      to a greater likelihood that the stacked vectors are linearly
	      dependent.

\end{enumerate}
