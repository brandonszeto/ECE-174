\begin{enumerate}[label=(\alph*)]
	\item Suppose that we are given $k$ vectors $\mathbf{u}_1, \mathbf{u}_2,
		      \ldots, \mathbf{u}_{\mathbf{k}} \in \mathbb{R}^n$. Now consider another
	      $m$ vectors $\mathbf{v}_1, \mathbf{v}_2, \ldots,
		      \mathbf{v}_{\mathbf{m}}$ such that
	      $\mathbf{v}_1, \mathbf{v}_2, \ldots, \mathbf{v}_{\mathbf{m}}
		      \in \operatorname{span}\left\{\mathbf{u}_{1}, \mathbf{u}_{2}, \ldots,
		      \mathbf{u}_{k}\right\}$. Show that $\operatorname{span}\left\{\mathbf{v}_{1},
		      \mathbf{v}_{2}, \ldots, \mathbf{v}_{m}\right\}$ is a subspace of
	      $\operatorname{span}\left\{\mathbf{u}_{1}, \mathbf{u}_{2}, \ldots,
		      \mathbf{u}_{k}\right\}$.


	      % SOLUTION
	      \par \textbf{Solution:}
	      \par A subspace is a subset of a vector space that contains the zero
	      vector, closed under addition, and closed under scalar multiplication.
	      Thus we must show the following: (i) $\operatorname{span}\left\{\mathbf{v}_{2},
		      \mathbf{v}_{2}, \ldots, \mathbf{v}_{m}\right\}$ is a subset of
	      $\operatorname{span}\left\{\mathbf{u}_{1}, \mathbf{u}_{2}, \ldots,
		      \mathbf{u}_{k}\right\}$ and that (ii) $\operatorname{span}\left\{\mathbf{v}_{2},
		      \mathbf{v}_{2}, \ldots, \mathbf{v}_{k}\right\}$ is a subspace.

	      \par (i) $\operatorname{span}\left\{\mathbf{v}_{2},
		      \mathbf{v}_{2}, \ldots, \mathbf{v}_{m}\right\}$ is a subset of
	      $\operatorname{span}\left\{\mathbf{u}_{1}, \mathbf{u}_{2}, \ldots,
		      \mathbf{u}_{k}\right\}$: Given that $\mathbf{v}_1, \mathbf{v}_2,
		      \ldots, \mathbf{v}_m$ are in the span of $\mathbf{u}_1, \mathbf{u}_2,
		      \ldots, \mathbf{u}_k$, this means that each $\mathbf{v}_i$ can be
	      expressed as a linear combination of the vectors $\mathbf{u}_1,
		      \mathbf{u}_2, \ldots, \mathbf{u}_k$. For each
	      $\mathbf{v}_i$, there exist coefficients $a_{i1}, a_{i2}, \ldots,
		      a_{ik}$ such that:
	      $$ \mathbf{v}_i = a_{i1}\mathbf{u}_1 + a_{i2}\mathbf{u}_2 + \ldots + a_{ik}\mathbf{u}_k $$
	      Now, consider an arbitrary vector $\mathbf{w}$ in the span of
	      $\mathbf{v}_1, \mathbf{v}_2, \ldots, \mathbf{v}_m$. We can express
	      $\mathbf{w}$ as a linear combination of the vectors $\mathbf{v}_1,
		      \mathbf{v}_2, \ldots, \mathbf{v}_m$. For each
	      $\mathbf{w}$, there exist coefficients $\beta_{i1}, \beta_{i2}, \ldots,
		      \beta_{ik}$ such that:
	      $$ \mathbf{w} = \beta_{i1}\mathbf{v}_1 + \beta_{i2}\mathbf{v}_2 + \ldots + \beta_{im}\mathbf{v}_m $$
	      Substituting the expression for $\mathbf{v}_i$ into the expression for $\mathbf{w}$, we get:
	      $$ \mathbf{w} = \beta_{i1}(a_{i1}\mathbf{u}_1 + a_{i2}\mathbf{u}_2 + \ldots + a_{ik}\mathbf{u}_k) + \beta_{i2}(a_{i1}\mathbf{u}_1 + a_{i2}\mathbf{u}_2 + \ldots + a_{ik}\mathbf{u}_k) + \ldots + \beta_{im}(a_{i1}\mathbf{u}_1 + a_{i2}\mathbf{u}_2 + \ldots + a_{ik}\mathbf{u}_k) $$
	      Distributing the coefficients, we get:
	      $$ \mathbf{w} = \sum_{i=1}^{m} \sum_{j=1}^{k} \beta_{ij}a_{ij}\mathbf{u}_j $$

	      This shows that the vector $\mathbf{w}$ is a linear combination of the
	      vectors $\mathbf{u}_1, \mathbf{u}_2, \ldots, \mathbf{u}_k$, which
	      means that $\mathbf{w}$ is in the span of these vectors.

	      \par (ii) $\operatorname{span}\left\{\mathbf{v}_{2},
		      \mathbf{v}_{2}, \ldots, \mathbf{v}_{k}\right\}$ is a subspace:
	      The span is defined as $\sum_{i=1}^{k} \alpha_i \mathbf{v}_i$,
	      where $\alpha_i \in \mathbb{R}$. In other words, the span is the set
	      of all linear combinations of the vectors $\mathbf{v}_1, \mathbf{v}_2,
		      \ldots, \mathbf{v}_{k}$. This satisfies the three properties of a
	      subspace: setting all $\alpha_i = 0$ demonstrates that the span
	      contains the zero vector and by the definition of linear combination,
	      any span must be closed under addition and scalar multiplication.

	      \par Together, this shows that $\operatorname{span}\left\{\mathbf{v}_{2},
		      \mathbf{v}_{2}, \ldots, \mathbf{v}_{k}\right\}$ is a subspace of
	      $\operatorname{span}\left\{\mathbf{u}_{1}, \mathbf{u}_{2}, \ldots,
		      \mathbf{u}_{k}\right\}$. \qed

	\item Given $\mathbf{A} \in \mathbb{R}^{n \times k}, \mathbf{B} \in
		      \mathbb{R}^{k \times m}$, show that $\mathcal{R}(\mathbf{C})$ is a
	      subspace of $\mathcal{R}(\mathbf{A})$ given that $\mathbf{C} = \mathbf{AB}$.

	      % SOLUTION BEGINS HERE
	      \par \textbf{Solution:}
	      \par To show that $\mathcal{R}(\mathbf{C})$ is a subspace of
	      $\mathcal{R}(\mathbf{A})$, we must show the following: (i)
	      $\mathcal{R}(\mathbf{C})$ is in $\mathcal{R}(\mathbf{A})$ and that
	      (ii) $\mathcal{R}(\mathbf{C})$ is a subspace.

	      \par (i) Writing out the matrix multiplication, we get:
	      $$
		      \left[\begin{array}{cccc}
				      a_{11}  & a_{12} & \cdots & a_{1 k} \\
				      a_{21}  & a_{22} &        & a_{2 k} \\
				      \vdots  &        & \ddots & \vdots  \\
				      a_{n 1} & a_{n2} & \cdots & a_{n k}
			      \end{array}\right]
		      \left[\begin{array}{cccc}
				      b_{11}  & b_{12}  & \cdots & b_{1 m} \\
				      b_{21}  & b_{22}  &        & b_{2 m} \\
				      \vdots  &         & \ddots & \vdots  \\
				      b_{k 1} & b_{k 2} & \cdots & b_{k m}
			      \end{array}\right]=
		      \left[\begin{array}{cccc}
				      c_{11}  & c_{12}  & \cdots & c_{1 m} \\
				      c_{21}  & c_{22}  &        & c_{2 m} \\
				      \vdots  &         & \ddots & \vdots  \\
				      c_{n 1} & c_{n 2} & \cdots & c_{n m}
			      \end{array}\right]
	      $$
	      It is clear that the matrix $\mathbf{A}$ consists of $k$ column
	      vectors of length $n$, and the matrix $\mathbf{C}$ consists of $m$
	      column vectors of length $n$. By
	      definition, $\mathcal{R}(\mathbf{A})$ is the span of its column
	      vectors. Since each column vector of $\mathbf{C}$ is a linear
	      combination of the column vectors of $\mathbf{A}$ (with coefficients
	      defined by $\mathbf{B}$), it follows that
	      the column vectors of $\mathbf{C}$ are in the span of the column
	      vectors of $\mathbf{A}$.

	      \par (ii) To show that $\mathcal{R}(\mathbf{C})$ is a subspace, we must
	      show that it satisfies the three properties of a subspace. To show
	      that $\mathcal{R}(\mathbf{C})$ contains the zero vector, we can
	      multiply all columns of $\mathbf{C}$ by the zero vector, which will
	      return the zero vector. To show that $\mathcal{R}(\mathbf{C})$ is
	      closed under addition, we can add two columns of $\mathbf{C}$ to get
	      a vector that is a linear combination of the columns of $\mathbf{C}$,
	      and is thus in the range of $\mathbf{C}$.
	      Finally, to show that $\mathcal{R}(\mathbf{C})$ is closed under scalar
	      multiplication, we can multiply a column of $\mathbf{C}$ by a scalar
	      to get a vector that is a linear combination of its columns and is
	      thus also in the range of $\mathbf{C}$.

	      \par Together, we have shown that $\mathcal{R}(\mathbf{C})$ is a subspace
	      of $\mathcal{R}(\mathbf{A})$. \qed

	\item Fill in the blank and justify your answer: Given the information in
	      part (b), $\mathcal{R}\left( \mathbf{C}^T \right)$ is a subspace of
	      $\mathcal{R}\left(\mathbf{B}^T \right)$.

	      % SOLUTION BEGINS HERE
	      \par \textbf{Solution:}
	      To determine the subspace that $\mathcal{R}\left( \mathbf{C}^T
		      \right)$ belongs to, let's examine its dimensions. From the definition
	      of transpose, we have:
	      $$
		      \mathbf{C}^T  =
		      \left[\begin{array}{cccc}
				      c_{11}  & c_{12}  & \cdots & c_{1 n} \\
				      c_{21}  & c_{22}  &        & c_{2 n} \\
				      \vdots  &         & \ddots & \vdots  \\
				      c_{m 1} & c_{m 2} & \cdots & c_{m n}
			      \end{array}\right]
	      $$
	      Where a given element $c_{ij} \in \mathbf{C}$ is now element $c_{ji}$.
	      Thus, we have that $\mathbf{C}^T \in \mathbb{R}^{m \times n}$. Now in
	      order to achieve a matrix in $\mathbb{R}^{n \times m}$ from our
	      original matrices $\mathbf{A}$ and $\mathbf{B}$, we must perform the
	      following operation:
	      $$
		      \left[\begin{array}{cccc}
				      b_{11}  & b_{12}  & \cdots & b_{1 k} \\
				      b_{21}  & b_{22}  &        & b_{2 k} \\
				      \vdots  &         & \ddots & \vdots  \\
				      b_{m 1} & b_{m 2} & \cdots & b_{m k}
			      \end{array}\right]
		      \left[\begin{array}{cccc}
				      a_{11}  & a_{12} & \cdots & a_{1 n} \\
				      a_{21}  & a_{22} &        & a_{2 n} \\
				      \vdots  &        & \ddots & \vdots  \\
				      a_{k 1} & a_{k2} & \cdots & a_{k n}
			      \end{array}\right]
		      =
		      \left[\begin{array}{cccc}
				      c_{11}  & c_{12}  & \cdots & c_{1 n} \\
				      c_{21}  & c_{22}  &        & c_{2 n} \\
				      \vdots  &         & \ddots & \vdots  \\
				      c_{m 1} & c_{m 2} & \cdots & c_{m n}
			      \end{array}\right]
	      $$

	      However, notice that matrices
	      $\mathbf{A}$ and $\mathbf{B}$ and have transposed and the order of
	      operations have changed such that:
	      $$ \mathbf{B}^T \mathbf{A}^T = \mathbf{C}^T $$
	      Therefore, following the same reasoning as in (b), we can conclude
	      that $\mathcal{R}\left( \mathbf{C}^T \right)$ is a subspace of
	      $\mathcal{R}\left(\mathbf{B}^T \right)$. \qed

\end{enumerate}
