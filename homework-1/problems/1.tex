\begin{enumerate}[label=(\alph*)]
	\item Given $k$ vectors $\mathbf{v}_1, \mathbf{v}_2, \ldots,
		      \mathbf{v}_{k} \in \mathbb{R}^n$, show that the
	      $\operatorname{span}\left\{\mathbf{v}_{2},
		      \mathbf{v}_{2}, \ldots, \mathbf{v}_{k}\right\}$ is a
	      subspace of $\mathbb{R}^n$.

	      % SOLUTION
	      \par \textbf{Solution:}
	      \par A subspace is a subset of a vector space that contains the zero
	      vector, closed under addition, and closed under scalar multiplication.
	      Thus we must show the following: (i) $\operatorname{span}\left\{\mathbf{v}_{2},
		      \mathbf{v}_{2}, \ldots, \mathbf{v}_{k}\right\}$ is a subset of
	      $\mathbb{R}^n$ and that (ii) $\operatorname{span}\left\{\mathbf{v}_{2},
		      \mathbf{v}_{2}, \ldots, \mathbf{v}_{k}\right\}$ is a subspace.

	      \begin{enumerate}[label=\roman*.]
		      \item $\operatorname{span}\left\{\mathbf{v}_{2},
			            \mathbf{v}_{2}, \ldots, \mathbf{v}_{k}\right\}$ is a subset of
		            $\mathbb{R}^n$: Because $\mathbf{v}_1, \mathbf{v}_2, \ldots,
			            \mathbf{v}_{k} \in \mathbb{R}^n$, we know that all possible linear
		            combinations of these vectors will also be in $\mathbb{R}^n$.
		            Therefore, $\operatorname{span}\left\{\mathbf{v}_{1},
			            \mathbf{v}_{2}, \ldots, \mathbf{v}_{k}\right\}$ is a subset of
		            $\mathbb{R}^n$.

		      \item $\operatorname{span}\left\{\mathbf{v}_{2},
			            \mathbf{v}_{2}, \ldots, \mathbf{v}_{k}\right\}$ is a subspace:
		            The span is defined as $\sum_{i=1}^{k} \alpha_i \mathbf{v}_i$,
		            where $\alpha_i \in \mathbb{R}$. In other words, the span is the set
		            of all linear combinations of the vectors $\mathbf{v}_1, \mathbf{v}_2,
			            \ldots, \mathbf{v}_{k}$. This satisfies the three properties of a
		            subspace: setting all $\alpha_i = 0$ demonstrates that the span
		            contains the zero vector:
		            $$
			            \begin{aligned}
				            \alpha_1\mathbf{v}_1 + \alpha_2\mathbf{v}_2 \cdots +
				            \alpha_k\mathbf{v}_k & = 0 \\
				            0 \cdot \mathbf{v}_1 + 0 \cdot \mathbf{v}_2 \cdots +
				            0 \cdot \mathbf{v}_k & = 0 \\
			            \end{aligned}
		            $$
		            Consider two sets of vectors $\mathbf{a}$ and $\mathbf{b}$:
		            $$
			            \begin{aligned}
				            (\mathbf{a}_1 + \mathbf{b}_1) +
				            (\mathbf{a}_2 + \mathbf{b}_2) \cdots +
				            (\mathbf{a}_k + \mathbf{b}_k) & = 0 \\
				            (\mathbf{a}_1 + \mathbf{a}_2 \cdots + \mathbf{a}_k) +
				            (\mathbf{b}_1 + \mathbf{b}_2 \cdots + \mathbf{b}_k)
				                                          & = 0 \\
				            0 + 0                         & = 0
			            \end{aligned}
		            $$
		            here, we have shown that the span is closed under addition.
		            Now consider some scalar $\alpha$:
		            $$
			            \begin{aligned}
				            \alpha(\mathbf{a}_1 + \mathbf{a}_2 \cdots + \mathbf{a}_k) & = 0 \\
				            \alpha \cdot 0                                            & = 0 \\
			            \end{aligned}
		            $$
		            Here, we have shown that the span is closed under scalar
		            multiplication.


	      \end{enumerate}

	      \par Together, this shows that $\operatorname{span}\left\{\mathbf{v}_{2},
		      \mathbf{v}_{2}, \ldots, \mathbf{v}_{k}\right\}$ is a subspace of
	      $\mathbb{R}^n$. \qed

	\item Given a matrix $A \in \mathbb{R}^{m \times n}$, show that the
	      $\mathcal{N}(\mathbf{A})$ is a subspace. Which vector space is it a
	      subspace of?

	      % SOLUTION
	      \par \textbf{Solution:}
	      \par The null space $\mathcal{N}(\mathbf{A})$ consists of all solutions $\mathbf{x}$
	      to the equation $\mathbf{A}\mathbf{x} = \mathbf{0}$, where $A \in
		      \mathbb{R}^{m \times n}$. In this equation, $\mathbf{x}$ must be a
	      column vector with $n$ elements, and is thus a subset of
	      $\mathbb{R}^n$. However, we must also prove

	      \begin{enumerate}[label=\roman*.]
		      \item $\mathcal{N}(\mathbf{A})$ contains the zero vector.
		            Trivially, when $\mathbf{x} = 0$
		            $\mathbf{Ax} = 0$
		      \item Now consider two non-trivial solutions $\mathbf{a}$ and
		            $\mathbf{b}$ that satisfy $\mathbf{Ax} = 0$:
		            $$
			            \begin{aligned}
				            \mathbf{A}(\mathbf{a} + \mathbf{b}) & = 0 \\
				            \mathbf{Aa} + \mathbf{Ab}           & = 0 \\
				            0 + 0                               & = 0
			            \end{aligned}
		            $$
		            Thus, $\mathcal{N}(\mathbf{A})$ is closed under addition.
		      \item Now consider some scalar $\alpha$:
		            $$
			            \begin{aligned}
				            \alpha\mathbf{Ax} & = 0 \\
				            \alpha \cdot 0    & = 0 \\
			            \end{aligned}
		            $$
	      \end{enumerate}
	      Together, this shows that $\mathcal{N}(\mathbf{A})$ is a subspace of
	      $\mathbb{R}^n$. \qed
\end{enumerate}
