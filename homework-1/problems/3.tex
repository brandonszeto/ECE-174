\begin{enumerate}[label=(\alph*)]
	\item Let $\mathbf{A}$ be a square matrix. Do $\mathbf{A}^2$ and
	      $\mathbf{A}$ always have the same nullspace?

	      % SOLUTION BEGINS HERE
	      \par \textbf{Solution:}
	      \par No, not necessarily. Consider the following example:
	      $$ \mathbf{A} = \left[\begin{array}{cc}
				      0 & 1 \\
				      0 & 0
			      \end{array}\right] $$
	      Therefore, we have
	      $$ \mathbf{A}^2 =
		      \left[\begin{array}{cc}
				      0 & 1 \\
				      0 & 0
			      \end{array}\right]
		      \left[\begin{array}{cc}
				      0 & 1 \\
				      0 & 0
			      \end{array}\right]
		      =
		      \left[\begin{array}{cc}
				      0 & 0 \\
				      0 & 0
			      \end{array}\right]
	      $$
	      Recall the definition of the nullspace of a matrix $\mathbf{A}$ to be
	      the set of vectors $\mathbf{x}$ such that $\mathbf{Ax} = \mathbf{0}$.
	      Consider the nullspace of the above matrices:
	      $$
		      \left[\begin{array}{cc}
				      0 & 1 \\
				      0 & 0
			      \end{array}\right]
		      \left[\begin{array}{c}
				      x_1 \\
				      x_2
			      \end{array}\right]
		      =
		      \left[\begin{array}{c}
				      0 \\
				      0
			      \end{array}\right]
	      $$
	      $$
		      \left[\begin{array}{cc}
				      0 & 0 \\
				      0 & 0
			      \end{array}\right]
		      \left[\begin{array}{c}
				      x_1 \\
				      x_2
			      \end{array}\right]
		      =
		      \left[\begin{array}{c}
				      0 \\
				      0
			      \end{array}\right]
	      $$
	      Clearly, the nullspace of $\mathbf{A}$ is the set of vectors where
	      $x_2 = 0$ while the nullspace of $\mathbf{A}^2$ is the set of vectors
	      where $x_1, x_2 \in \mathbb{R}$. Thus we have demonstrated that the
	      set of vectors in the nullspace of $\mathbf{A}^2$ is not necessarily
	      the same as the nullspace of $\mathbf{A}$. \qed

	\item Given $\mathbf{A} \in \mathbb{R}^{n \times n}$, if
	      $\mathcal{N}(\mathbf{A})$ contains only the zero vector, what vectors
	      are in the nullspace of $\mathbf{B}=\left[\begin{array}{lll}
				      \mathbf{A} & \mathbf{A} & \mathbf{A}
			      \end{array}\right]$ ?

	      % SOLUTION BEGINS HERE
	      \par \textbf{Solution:}
	      \par $\mathcal{N}(\mathbf{B})$ will contain only
	      the zero vector. Because $\mathcal{N}(\mathbf{A})$ contains
	      only the zero vector, only the zero vector $\mathbf{x} = 0$ satisfies
	      $\mathbf{A}\mathbf{x} = \mathbf{0}$. Expanding $\mathbf{B}$, we have:
	      $$
		      \mathbf{Bx} =
		      \left[\begin{array}{lll}
				      \mathbf{A} &
				      \mathbf{A} &
				      \mathbf{A}
			      \end{array}\right]
		      \left[\begin{array}{l}
				      x_1 \\
				      x_2 \\
				      x_3
			      \end{array}\right] =
		      \mathbf{Ax_1} +
		      \mathbf{Ax_2} +
		      \mathbf{Ax_3} = 0
	      $$
	      Thus, $\mathcal{N}(\mathbf{B})$ contains any combination of
	      vectors in $\mathcal{N}(\mathbf{A})$ that sum to the zero vector.
	      However, because $\mathcal{N}(\mathbf{A})$ contains only the zero
	      vector, $\mathcal{N}(\mathbf{B})$ must also contain only the zero.
	      \qed

	\item How is the nullspace of $\mathbf{C}$ related to the nullspace of
	      $\mathbf{A}$ and $\mathbf{B}$, if $\mathbf{C}=\left[\begin{array}{l}
				      \mathbf{A} \\
				      \mathbf{B}
			      \end{array}\right]$?

	      % SOLUTION BEGINS HERE
	      \par \textbf{Solution:}
	      \par Recall the definition of the nullspace of a
	      matrix $\mathbf{C}$ to be the set of vectors $\mathbf{x}$ such that
	      $\mathbf{Cx} = \mathbf{0}$. Considering the definition of $\mathbf{C}$
	      above, we have
	      $$
		      \mathbf{Cx} =
		      \left[\begin{array}{l}
				      \mathbf{A} \\
				      \mathbf{B}
			      \end{array}\right]
		      \mathbf{x} =
		      \left[\begin{array}{l}
				      \mathbf{Ax} \\
				      \mathbf{Bx}
			      \end{array}\right] = 0
	      $$
	      Therefore, for $\mathbf{Cx} = 0$ to hold, $\mathbf{x}$ must belong to
	      the nullspace of both $\mathbf{A}$ and $\mathbf{B}$.
	      \qed

\end{enumerate}
